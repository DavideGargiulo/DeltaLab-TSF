\chapter{Conclusioni e Sviluppi Futuri}

\section{Obiettivi Raggiunti}

Il progetto ha raggiunto con successo tutti gli obiettivi prefissati:
\begin{itemize}
  \item Implementazione completa di un'applicazione client-server funzionante;
  \item Interfaccia utente intuitiva e responsive realizzata con JavaFX;
  \item Backend robusto e performante implementato in C con Mongoose;
  \item Sistema di autenticazione sicuro con password hashate;
  \item Comunicazione in tempo reale tramite WebSocket;
  \item Supporto multilingua attraverso integrazione con LibreTranslate;
  \item Containerizzazione completa con Docker per facilità di deployment;
  \item Gestione persistente dei dati con PostgreSQL.
\end{itemize}

\section{Difficoltà Incontrate}

Durante lo sviluppo del progetto sono state affrontate diverse sfide:
\begin{itemize}
  \item \textbf{Sincronizzazione}: garantire che tutti i client visualizzino lo stesso stato della partita in tempo reale ha richiesto un'attenta gestione dei messaggi WebSocket;
  \item \textbf{Gestione delle disconnessioni}: implementare un sistema robusto per gestire disconnessioni improvvise senza compromettere l'esperienza degli altri giocatori;
\end{itemize}

\newpage

\section{Considerazioni Finali}

Questo progetto ha rappresentato un'opportunità significativa per applicare concetti di programmazione di sistema, networking e sviluppo di applicazioni distribuite. L'integrazione di tecnologie diverse (C, Java, PostgreSQL, Docker) ha permesso di acquisire esperienza pratica nella gestione di un sistema complesso multi-componente.

L'utilizzo di Docker ha semplificato notevolmente il processo di deployment, garantendo coerenza tra ambienti di sviluppo e produzione. Questa esperienza ha evidenziato l'importanza della containerizzazione nelle moderne architetture applicative.

Il supporto multilingua attraverso LibreTranslate ha dimostrato come l'integrazione con servizi esterni possa arricchire significativamente le funzionalità di un'applicazione, rendendola accessibile a un pubblico più ampio.

In conclusione, il progetto ha raggiunto tutti gli obiettivi prefissati, risultando in un'applicazione funzionale che offre un'esperienza di gioco divertente e coinvolgente. Le basi create permettono facilmente l'aggiunta di nuove funzionalità e miglioramenti futuri.

\section{Ringraziamenti}

Si ringraziano tutti i membri del team per il contributo allo sviluppo del progetto, per la collaborazione durante le fasi di progettazione e implementazione, e per il supporto reciproco nel superare le difficoltà incontrate. Un ringraziamento particolare va ai docenti del corso di Laboratorio di Sistemi Operativi per le competenze trasmesse e per il supporto fornito durante lo sviluppo del progetto.

Si ringrazia inoltre De Gregorio Gennaro per aver fornito il suo raspberry pi per l'hosting del server.