\chapter{Logica di Gioco}

\section{Regole del Telefono Senza Fili}

Il gioco del telefono senza fili è un gioco collaborativo in cui i giocatori si alternano nello scrivere frasi, creando una catena di interpretazioni che spesso porta a risultati divertenti e inaspettati.

Le regole del gioco sono le seguenti:
\begin{enumerate}
  \item Il primo giocatore scrive una frase iniziale;
  \item Il secondo giocatore visualizza la frase e deve continuarla;
  \item Il terzo giocatore visualizza il complessivo e deve continuare a sua volta;
  \item Il processo continua fino a quando tutti i giocatori hanno partecipato;
  \item Al termine, viene mostrata la sequenza completa per vedere come la frase si è trasformata.
\end{enumerate}

\section{Fasi della Partita}

Una partita attraversa diverse fasi, gestite dal server:

\subsection{Fase di Attesa}
La partita rimane in fase di attesa finché non si raggiunge il numero minimo di giocatori richiesto. Durante questa fase, i giocatori possono entrare e uscire liberamente dalla lobby. Il creatore della lobby può decidere di avviare la partita manualmente quando il numero minimo di giocatori è raggiunto (minimo 4 giocatori).

\subsection{Fase di Inizializzazione}
Quando la partita viene avviata, il server:
\begin{itemize}
  \item Blocca l'ingresso di nuovi giocatori attivi (gli spettatori possono ancora unirsi);
  \item Invia a tutti i client le informazioni iniziali della partita;
\end{itemize}

\newpage

\subsection{Fase di Gioco}
Durante ogni turno:
\begin{itemize}
  \item Un giocatore riceve la frase dal turno precedente o, se è il primo turno, scrive la frase iniziale;
  \item Una volta completato, il contenuto viene inviato al server;
  \item Il server passa la frase al giocatore successivo;
  \item Gli altri giocatori e gli spettatori vedono un'indicazione di chi sta giocando.
\end{itemize}

\subsection{Fase Finale}
Quando tutti i giocatori hanno completato i loro turni, la partita termina e viene mostrata la sequenza completa di frasi.

\section{Gestione dei Turni}

Il server mantiene lo stato di ogni partita in una struttura dati che include:
\begin{itemize}
  \item L'elenco dei giocatori e il loro ordine;
  \item L'indice del giocatore corrente;
  \item La catena di contenuti (frasi) prodotti finora;
\end{itemize}

Quando un giocatore completa il suo turno, il server:
\begin{enumerate}
  \item Lo aggiunge alla catena;
  \item Incrementa l'indice del turno;
  \item Notifica il giocatore successivo tramite WebSocket;
\end{enumerate}

\section{Gestione delle Disconnessioni}

Il sistema è progettato per gestire le disconnessioni improvvise dei giocatori. Quando un client si disconnette:
\begin{itemize}
  \item Il server rileva la disconnessione tramite l'evento \texttt{onClose} del WebSocket;
  \item Se è il turno del giocatore disconnesso, viene saltato automaticamente;
  \item Gli altri giocatori vengono notificati della disconnessione;
  \item Se il numero di giocatori attivi scende sotto il minimo richiesto, la partita continua fino a quando i giocatori rimasti hanno partecipato.
\end{itemize}