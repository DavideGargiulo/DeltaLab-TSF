\chapter{Implementazione del Backend}

\section{Server C e Mongoose}

Il server è implementato in C utilizzando la libreria Mongoose, che fornisce un framework leggero e efficiente per la gestione delle connessioni HTTP e WebSocket. Il file \textbf{server.c} rappresenta il punto di ingresso dell'applicazione e si occupa dell'inizializzazione del server, della configurazione delle route e della gestione del ciclo di vita dell'applicazione.

All'avvio, il server esegue le seguenti operazioni:
\begin{itemize}
  \item Configura il gestore di eventi Mongoose per processare le richieste HTTP e WebSocket;
  \item Registra le route per le varie operazioni (registrazione, login, gestione lobby);
  \item Avvia il loop principale che rimane in ascolto delle connessioni in arrivo.
\end{itemize}

La comunicazione tra client e server avviene tramite messaggi JSON strutturati, che vengono parsati e processati dal server. Ogni messaggio contiene un campo \texttt{type} che identifica il tipo di operazione richiesta e eventuali parametri necessari per l'esecuzione dell'operazione.

\section{Gestione del Database}

La connessione al database PostgreSQL viene stabilita quando è necessario utilizzare il database. Il modulo di gestione del database espone funzioni per eseguire query SQL e recuperare i risultati in modo sicuro, prevenendo attacchi di SQL injection attraverso l'uso di prepared statements.

Le principali operazioni sul database includono:
\begin{itemize}
  \item \textbf{Inserimento utenti}: quando un nuovo utente si registra, i suoi dati vengono inseriti nella tabella \texttt{users} con la password hashata utilizzando un algoritmo di hashing sicuro;
  \item \textbf{Autenticazione}: durante il login, viene verificata la corrispondenza tra username e password hashata;
  \item \textbf{Gestione lobby}: creazione, aggiornamento e cancellazione delle lobby nella tabella \texttt{lobby};
  \item \textbf{Associazione giocatori}: gestione delle relazioni molti-a-molti tra utenti e lobby attraverso la tabella \texttt{lobby\_players}.
\end{itemize}

\section{Controller e Routing}

La cartella \textbf{controllers} contiene i file che implementano la logica di routing delle richieste HTTP. Ogni controller è responsabile di gestire un insieme specifico di operazioni correlate. I principali controller implementati sono:

\begin{itemize}
  \item \textbf{auth\_controller.c}: gestisce le operazioni di autenticazione, incluse registrazione e login;
  \item \textbf{lobby\_controller.c}: gestisce la creazione, l'aggiornamento e la cancellazione delle lobby;
  \item \textbf{user\_controller.c}: gestisce le operazioni relative agli utenti, per recuperare le informazioni dell'utente.
\end{itemize}

Ogni controller implementa funzioni che ricevono come parametro la connessione Mongoose e il messaggio HTTP, estraggono i parametri necessari, invocano la logica di business appropriata e restituiscono una risposta JSON al client.