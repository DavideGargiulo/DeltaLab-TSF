\chapter{Introduzione}

Questo progetto consiste nella realizzazione di un'applicazione client-server per poter permette di giocare con altri client al gioco del \textbf{telefono senza fili}.

Le tecnologie richieste per quest'applicazione sono le seguenti:
\begin{itemize}
  \item \textbf{C}: linguaggio per il backend;
  \item \textbf{Java}: linguaggio scelto per il frontend;
  \item \textbf{JavaFx}: libreria per la realizzazione dell'interfaccia grafica in Java;
  \item \textbf{Mongoose}: libreria per la gestione delle connessioni WebSocket in C;
  \item \textbf{Docker}: strumento per la containerizzazione dell'applicazione;
  \item \textbf{Maven}: strumento di gestione delle dipendenze per Java;
  \item \textbf{PostgreSql}: sistema di gestione di database relazionali.
\end{itemize}

Ogni client può svolgere determinate azioni come:
\begin{itemize}
  \item Registrarsi al server;
  \item Effettuare il login;
  \item Creare una nuova partita;
  \item Unirsi a una partita esistente;
  \item Unirsi a una partita già in corso come spettatore;
  \item Giocare.
\end{itemize}

Il server, invece, si occupa di:
\begin{itemize}
  \item Gestire le connessioni dei client;
  \item Mantenere lo stato delle partite;
  \item Gestire la logica di gioco;
  \item Comunicare con i client tramite WebSocket.
\end{itemize}