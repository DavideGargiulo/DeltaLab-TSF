\chapter{Implementazione del Frontend}

\section{Architettura JavaFX}

Il frontend dell'applicazione è sviluppato in JavaFX, un framework moderno per la creazione di interfacce grafiche in Java. L'architettura segue il pattern MVC (Model-View-Controller), dove le viste sono definite in file FXML, i controller gestiscono la logica dell'interfaccia e i modelli rappresentano i dati.

La struttura dell'applicazione JavaFX è organizzata come segue:
\begin{itemize}
  \item \textbf{View}: file FXML che definiscono la struttura e il layout delle schermate;
  \item \textbf{Controller}: classi Java che gestiscono gli eventi dell'interfaccia utente e coordinano l'interazione con il backend;
  \item \textbf{Model/DTO}: classi che rappresentano i dati trasferiti tra frontend e backend;
  \item \textbf{Utils}: classi di utilità per funzionalità comuni come la gestione WebSocket e la traduzione.
\end{itemize}

\section{Schermate Principali}

L'applicazione è composta da diverse schermate, ciascuna con una specifica funzionalità:

\subsection{Schermata di Login}
La schermata di login permette all'utente di autenticarsi inserendo username e password. Il controller associato valida i dati inseriti e invia una richiesta HTTP POST al server con le credenziali. In caso di successo, l'utente viene reindirizzato alla schermata principale; in caso di errore, viene visualizzato un messaggio appropriato.

\subsection{Schermata di Registrazione}
La schermata di registrazione consente a nuovi utenti di creare un account. Il controller verifica che i campi siano compilati correttamente, che la password soddisfi i requisiti minimi di sicurezza. Successivamente, invia una richiesta al server per creare il nuovo utente.

\subsection{Schermata Lobby}
La schermata lobby mostra l'elenco delle partite disponibili e permette all'utente di:
\begin{itemize}
  \item Visualizzare le lobby esistenti con informazioni sul numero di giocatori e lo stato;
  \item Creare una nuova lobby specificando il senso di rotazione;
  \item Unirsi a una lobby esistente come giocatore;
  \item Unirsi a una partita in corso come spettatore.
\end{itemize}

La lista delle lobby viene aggiornata tramite un bottone dedicato, che invia richieste HTTP al server, garantendo che l'utente visualizzi sempre informazioni aggiornate.

\subsection{Schermata di Gioco}
La schermata di gioco rappresenta l'interfaccia principale durante una partita. Gli elementi principali includono:
\begin{itemize}
  \item Area di visualizzazione della frase corrente;
  \item Barra per scrivere, a seconda del turno;
  \item Lista dei giocatori partecipanti;
\end{itemize}

\section{Gestione WebSocket}

La classe \textbf{LobbyWebSocketClient} estende la classe \texttt{WebSocketClient} fornita dalla libreria Java-WebSocket e implementa la logica per gestire la comunicazione in tempo reale con il server.

I principali metodi implementati sono:
\begin{itemize}
  \item \texttt{onOpen()}: invocato quando la connessione WebSocket viene stabilita con successo;
  \item \texttt{onMessage()}: riceve i messaggi dal server e li processa in base al tipo;
  \item \texttt{onClose()}: gestisce la chiusura della connessione;
  \item \texttt{onError()}: gestisce eventuali errori durante la comunicazione;
  \item \texttt{send()}: invia messaggi JSON al server.
\end{itemize}

Durante una partita, il WebSocket viene utilizzato per:
\begin{itemize}
  \item Ricevere aggiornamenti in tempo reale sullo stato della partita;
  \item Inviare le azioni del giocatore (frasi scritte);
  \item Sincronizzare i turni tra i giocatori;
  \item Notificare l'ingresso o l'uscita di giocatori.
\end{itemize}